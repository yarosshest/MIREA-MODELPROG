\graphicspath{{./img}} % path to graphics

\sextion{Постановка задачи}
Смоделировать работу сервера, основываясь на выполненной работе №9.


\section*{Выполнение практической работы}
\addcontentsline{toc}{section}{Выполнение практической работы}
\subsection{Изменение текущей модели}
Откроем предыдущую модель и удалим из нее некоторые параметры и графики.
Добавим два параметра timeMean, коэффициент и переменную время выполнения.
timeMean по умолчанию зададим как 29 типа int, коэффициент зададим как 1 типа double, время выполнения зададим как double.

\begin{image}
	\includegraphics[width=1\textwidth]{img1}
	\caption{Добавление новых параметров}
	\label{fig:model:res}
\end{image}

Также выделим объект sink и в поле Действие при входе установим следующий код:
if( sink.in.count() == количествоЗапросов ){ времяВыполнения = time(); stopSimulation();}

\clearpage

\subsection{Создание нового эксперимента}
Создадим новый эксперимент типа варьирования параметров.
В свойствах укажем максимальный размер памяти – 256мб, а количество прогонов – 9604.
Также добавим параметры и зададим их значение. В разделе Действия Java установим код data.reset();

\begin{image}
	\includegraphics[width=1\textwidth]{img2}
	\caption{Создание эксперимента}
	\label{fig:data:res}
\end{image}

\clearpage

\subsection{Создание интерфейса}
Создадим интерфейс, на котором будет изображена гистограмма, ее данные и данные с ее средним значением.

\begin{image}
	\includegraphics[width=1\textwidth]{img3}
	\caption{Создание эксперимента}
	\label{fig:interface}
\end{image}

\clearpage

\section*{\LARGE Вывод}
\addcontentsline{toc}{section}{Вывод}
В ходе выполнения данной практической работы была выполнена обратная задача путем создания модели сервера.
