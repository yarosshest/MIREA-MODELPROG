\graphicspath{{./img}} % path to graphics

\section*{Выполнение практической работы}
\addcontentsline{toc}{section}{Выполнение практической работы}
Агентное моделирование --- самый молодой из основных методов
моделирования. Тем не менее некоторые агентные модели можно
встретить в старых журналах и книгах, таких как публикация Томаса
Шеллинга "<Микромотивы и макроповедение"> 1978 года выпуска (Schelling,
1978). Эти труды можно рассматривать как зарождение концепции
агентного моделирования. Десятилетиями его практическое применение
было ограничено уровнем развития компьютерной техники. В 90-х агентное
моделирование оставалось чисто научной темой, но в 21-м веке с расцветом
компьютерной техники оно стало коммерчески применимо для решения
крупномасштабных бизнес-задач, более того, теперь его применяют все
чаще по сравнению с другими методами.\par
Агентное моделирование и системная динамика --- диаметрально
противоположные методы. В то время как системная динамика основана
на целостном описании процесса, агентное моделирование использует
принцип от частного к общему, который выражается в объединенном
поведении системы, складывающемся из совокупности поведения объектов
(агентов). Результирующее поведение системы представляет собой
совокупность действий множества отдельных агентов.\par
Основной объект агентной модели очень прост, поскольку включает лишь
пару популяций агентов. Популяция агентов --- это множество агентов,
которую также называют набором или коллекцией. В нашем случае
это разработчики и стажеры. Событие найм периодически добавляет
новых Стажеров в систему.\par
Блок-схема агента разработчик проиллюстрированна
на рисунке~\ref{fig:agent:develop}.

\begin{image}
	\includegrph{2023-05-23\_08-46-56}
	\caption{Агент разработчик}
	\label{fig:agent:develop}
\end{image}

Блок-схема агента стажер проиллюстрированна
на рисунке~\ref{fig:agent:stag}.

\begin{image}
	\includegrph{2023-05-23\_08-47-09}
	\caption{Агент стажер}
	\label{fig:agent:stag}
\end{image}

Если мы запустим модель, то увидим, что проект
выполняется за 83 дня (рис. \ref{fig:model:res}).

\begin{image}
	\includegrph{2023-05-23\_08-47-30}
	\caption{Запуск модели разработки ПО}
	\label{fig:model:res}
\end{image}

\clearpage

\section*{\LARGE Вывод}
\addcontentsline{toc}{section}{Вывод}
В данной практической работе была создана модель приозводства ПО,
исользуя метод --- агентное моделирование.

