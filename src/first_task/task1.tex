\section*{\LARGE Введение}
\addcontentsline{toc}{section}{Введение}
Проблемы, связанные с отысканием наилучших решений для
достижения поставленных целей при ограниченных возможностях
(ресурсах), вставали перед людьми всегда. Концепция принятия ре-
шения в качестве первичного элемента деятельности рассматривает
решение как сознательный выбор одного из ряда вариантов.\par
Применение математических методов предполагает построение
математической модели объекта анализа. При построении модели
ситуации принятия решения дается формализованное описание до-
ступных вариантов действий и возможных последствий их реализа-
ции. При этом особое внимание уделяется выявлению и описанию
предпочтений ЛПР (\textit{лицо принимающее решение}).
Его цели чаще всего моделируются стремлением
к увеличению или же уменьшению специальных функций, называе-
мых критериями (а также показателями эффективности или качества,
целевыми функциями).\par
Принципиальная сложность многокритериальных задач состоит в
том, что обычно не существует варианта, который был бы наилучшим
сразу по всем критериям: если по одному из критериев вариант очень
хорош, то по другому, как правило, он будет далеко не лучший.

\newpage

\section*{\LARGE Выполнение практической работы}
\addcontentsline{toc}{section}{Выполнение практической работы}

\section{Постановка задачи}
Провести оценку качества программного обеспечения и осуществить
обоснованный выбор варианта программного обеспечения, применяемого в
профессиональной сфере в соответствии с указанной методикой.

\section[Выбор категорий]{Выбор категории анализируемого программного
обеспечения}
В качестве категории для анализа выбраны видеоредакторы:
\begin{itemize}
    \item VSDC;
    \item Kapwing;
    \item iMovie;
    \item Kdenlive;
    \item Lightworks;
    \item Avid Media Composer First;
    \item Openshot;
    \item Clipchamp;
    \item Videopad;
    \item Davinci Resolve.
\end{itemize}

\section{Характеристики и атрибуты программного
обеспечения}
В соответствии с Национальным стандартом РФ ГОСТ Р ИСО/МЭК 25010-2015
“Информационные технологии” были указаны следующие
характеристики и атрибуты программного обеспечения:
\begin{itemize}
    \item функциональная пригодность;
    \item уровень производительности;
    \item совместимость;
    \item удобство использования;
    \item надёжность;
    \item защищённость;
    \item сопровождаемость;
    \item стоймость.
\end{itemize}

\section[Определение занчений]{Определение значения характеристик
и атрибутов анализируемого программного обеспечения}
Были выделены следующие определения в рассматриваемой области
программного обеспечения:
\begin{itemize}
    \item Уровень производительности --- скорость работы ПО;
    \item Совместимость --- степень зависимости от оборудования;
    \item Удобство использования --- наличие комбинаций горячих клавиш,
    их количество и понятность, возможность устанавливать плагины;
    \item Функциональная пригодность --- способность ПО решать нужный
    набор задач;
    \item Сопровождаемость --- длительность и качество поддержки ПО;
    \item Переносимость (мобильность) --- степень многоплатформенности ПО.
\end{itemize}

Далее определили и нормализовали характеристики программного обеспечения,
данные приведены в таблице~\ref{table:specifications}.

\begin{table}[h!tp]
    \centering
    \resizebox{\textwidth}{!}{\begin{tabular}{|l|l|l|l|l|l|l|l|l|}
        \toprule
        \textbf{Вес} & \textbf{0,10} & \textbf{0,10} & \textbf{0,20} & \textbf{0,10} & \textbf{0,20} & \textbf{0,10} & \textbf{0,10} & \textbf{0,10} \\ \hline
        Редактор & функциональная пригодность & уровень производительности & совместимость & удобство использования & надёжность & защищённость & сопровождаемость & стоймость \\ \hline
        VSDC & 0,93 & 0,43 & 0,90 & 0,91 & 0,12 & 0,66 & 0,49 & 0,92 \\ \hline
        Kapwing & 0,04 & 0,48 & 0,64 & 0,51 & 0,36 & 0,14 & 0,75 & 0,80 \\ \hline
        iMovie & 0,74 & 0,16 & 0,40 & 0,83 & 0,57 & 0,61 & 0,68 & 0,62 \\\hline
        Kdenlive & 0,20 & 0,77 & 0,48 & 0,22 & 0,28 & 0,80 & 0,83 & 0,60 \\ \hline
        Lightworks & 0,06 & 0,23 & 0,43 & 0,14 & 0,96 & 0,02 & 0,79 & 0,23 \\ \hline
        Avid Media Composer First & 0,77 & 0,42 & 0,47 & 0,31 & 0,89 & 0,87 & 0,82 & 0,30 \\ \hline
        Openshot & 0,30 & 0,85 & 0,78 & 0,89 & 0,12 & 0,42 & 0,22 & 0,17 \\ \hline
        Clipchamp & 0,18 & 0,74 & 0,78 & 0,41 & 0,35 & 0,83 & 0,95 & 0,18 \\ \hline
        Videopad & 0,84 & 0,99 & 0,00 & 0,05 & 0,68 & 0,70 & 0,23 & 0,75 \\ \hline
        Davinci Resolve & 0,22 & 0,25 & 0,93 & 0,32 & 0,50 & 0,99 & 0,47 & 0,09 \\ \hline
        \bottomrule
    \end{tabular}}
    \label{tab:}
\end{table}


В качестве веса каждой из характеристик были составлены следующие
данные:
\begin{itemize}
    \item функциональная пригодность --- 0.1;
    \item уровень производительности --- 0.1;
    \item совместимость --- 0.2;
    \item удобство использования --- 0.1;
    \item надёжность --- 0.2;
    \item защищённость --- 0.1;
    \item сопровождаемость --- 0.1;
    \item стоймость --- 0.1.
\end{itemize}

\section{Анализ методов многокритериального принятия решений}
В качестве методов были выбраны следующие:
\begin{itemize}
    \item SAW;
    \item TOPSIS;
    \item MOUT.
\end{itemize}
\subsection{Метод SAW}
Метод SAW (Simple Additive Weighting), или метод простного аддитивного
взвешивания, является одним из самых известных и широко используемых
методов многокритериального принятия решений.

Он заключается в том, чтобы количественно измерить значимость
критериев по каждому альтернативному варианту, построив на этой основе
матрицу реше-ний A, из которой получают нормализованную матрицу решений R,
определяющую значимость (вес) каждого критерия, из которой, в свою очередь,
выводится обобщен-ная оценка каждой представленной альтернативы.

В целом процесс нахождения наилучшего решения может быть разделен на
следующие этапы:
\begin{itemize}
    \item анализ поставщиков по критериям;
    \item определение весов критериев;
    \item нормирование критериев;
    \item определение рейтинга поставщиков путем умножения
    значений критериев на веса.
\end{itemize}

Алгоритм расчета методом SAW следующий:
\begin{enumerate}
    \item Пусть \(C = \{c_1, c_2, \ldots, c_n\}\) --- множество
    оцениваемых критериев, \(A~=~\{a_1, a_2, \ldots, a_n\}\) --- множество
    потенциальных поставщиков.
    \item Нахождение нормированных значений матрицы оценок критериев.
    Для нормирования матрицы оценок критериев находятся наилучшие
    значения \(x_{ij}\) исходной матрицы значений критериев
    \(X = \{x_{ij}\}\), где – значение критерия \(c_i\)
    c множества \(C=\{c_i\}\) для поставщика \(a_i\) множества
    \(A = \{a_j\}\), по следующим формулам:\\
    \(\bar x_{ij} = \frac{x_{ij}}{max\;x_{ij}}\),
    если показатели эффективности максимизируются.\\
    \(\bar x_{ij} = \frac{min\;x_{ij}}{x_{ij}}\),
    если показатели эффективности максимизируются.
    \item Для более объективного результата, вводятся коэффициенты веса
    \(w_i \in [0,1]\) каждому критерию. Данные коэффициенты позволяют
    провести оценку с учетом приоритетности и весомости критериев.
    Сумма коэффициентов удельного веса всех критериев равна.
\end{enumerate}

Таким образом, расчет рейтинга поставщиков в общем виде можно
представить в виде выражения:
\[V_i = \sum_{j=0}^{n} w_j r_{ij}\]

\subsection{Метод TOPSIS}
Это метод компенсационного агрегирования, который сравнивает набор
альтернатив, нормализует оценки по каждому критерию и вычисляет
геометрическое расстояние между каждой альтернативой и идеальной
альтернативой, которая является лучшим результатом по каждому критерию.
Веса критериев в методе TOPSIS могут быть рассчитаны с использованием
подхода с порядковым приоритетом, аналитического иерархического процесса и т.д.

Предполагается, что критерии TOPSIS монотонно увеличиваются или уменьшаются.
Обычно требуется нормализация, поскольку параметры или критерии часто имеют
несовместимые размеры в многокритериальных задачах.
Компенсационные методы, такие как TOPSIS, допускают компромиссы между
критериями, когда плохой результат по одному критерию может быть сведен
на нет хорошим результатом по другому критерию. Это обеспечивает более
реалистичную форму моделирования, чем некомпенсационные методы,
которые включают или исключают альтернативные решения, основанные на
жестких ограничениях.

Алгоритм метода TOPSIS:
\begin{enumerate}
    \item Пусть \(C = \{c_i\}\) --- множество оцениваемых критериев,
    \(A = \{a_j\}\) --- множество потенциальных поставщиков,
    на основании которых строится матрица значений критериев
    \(X = (x_{ij})\).\par
    Для получения матрицы нормированных значений критериев
    \(P = (p_{ij})\), критерии переводятся в безразмерный вид по формуле:
    \[p_{ij}=\frac{x_{ij}}{\sqrt{\sum_{j=1}^{n}(x_{ij}^2)}}\]
    \item Затем строится матрица взвешенных значений критериев,
    где коэффициенты веса \(w_i \in [0,1]\). Матрицу нормализованных
    взвешенных значений можно представить в виде:
    \[\tilde P=(w_i p_{ij})=(\tilde p_{ij})\]
    \item На следующем этапе находятся идеально-позитивное
    и идеальнонегативное решения.
    \[A^+=(max(\tilde p_{11}), \ldots, max(\tilde p_{1n}))
    =(\tilde p_j^+, \ldots, \tilde p_n^+)\]
    \[A^-=(min(\tilde p_{11}), \ldots, min(\tilde p_{1n}))
    =(\tilde p_j^-, \ldots, \tilde p_n^-)\]
    \item Последним шагом будет нахождение относительной близости к
    идеально-позитивному решению по формуле:
    \[P_j^+=\frac{S_{j}^-}{S_j^+ + S_j^-}\]
    \item Выбирается альтернатива, для которой значение относительной
    близости будет ближе к 1.
\end{enumerate}

\subsection{Метод Mout}
Научный подход в принятии решений, известный как теория многокритериальной
полезности (Multi-Attribute Utility Theory MAUT),
отличают следующие особенности:
\begin{itemize}
    \item для каждой альтернативы \textit{A(i)} строится функция
    полезности \textit{U(A(i))}, имеющая аксиоматическое
    (чисто математическое) обоснование;
    \item некоторые условия, определяющие форму этой функции,
    подвергаются проверке в диалоге с ЛПР;
    \item обычно решается задача построения решающего правила для любых
    гипотетически возможных альтернатив, а полученные результаты
    используются для оценки заданных альтернатив.
\end{itemize}

Точно так же, как и классическая теория полезности, MAUT имеет
аксиоматическое обоснование. Это означает, что формулируются
некоторые условия (аксиомы), которым должна удовлетворять функция
полезности ЛПР.\par
Основным достоинством подхода MAUT является строгое математическое
обоснование вида функции полезности. Подход аналитической иерархии чисто
эмпирический, но его отличает простота и направленность па сравнение
заданного множества альтернатив.

Этапы решения задачи при подходе MAUT:
\begin{enumerate}
    \item Разработать перечень критериев.
    \item Построить функции полезности по каждому из критериев.
    \item Проверить некоторые условия, определяющие вид общей
    функции полезности.
    \item Построить зависимость между оценками альтернатив
    по критериям и общим качеством альтернативы
    (многокритериальная функция полезности).
    \item Оценить вес имеющиеся альтернативы и выбрать лучшую из них.
\end{enumerate}

\section{Решение}
Далее были реализованы рассматриваемые методы на языке программирования
Python.

Код метода SAW:
\begin{lstlisting}[language=Python
, caption=\leftline{doc.py}
, label=lst:doc]
def saw(w, apps):
	res = []
	for i in range(len(apps)):
		s = 0
		for j in range(len(apps[0])):
			s += apps[i][j] * w[j]
		res.append(round(s, 3))
	return res
\end{lstlisting}

Код метода MOUT:
\begin{lstlisting}[language=Python]
def mout(w, apps):
	mx = copy.deepcopy(apps)
	x_max = []
	x_min = []
	for i in range(len(mx[0])):
		vec = []
		for x in mx:
			vec.append(x[i])
		x_max.append(max(vec))
		x_min.append(min(vec))
	for i in range(len(mx)):
		for j in range(len(mx[0])):
			mx[i][j] = (mx[i][j]-x_min[j])
			mx[i][j] /= (x_max[j]-x_min[j])
			mx[i][j] = round(mx[i][j], 3)
	return saw(w, mx)
\end{lstlisting}

Код метода TOPSIS:
\begin{lstlisting}[language=Python]
def topical(w, apps):
	mx = copy.deepcopy(apps)
	x_sqrt = []
	for j in range(len(mx[0])):
		x = 0
		for i in range(len(mx)):
			x += mx[i][j] ** 2
		x_sqrt.append(math.sqrt(x))
	for i in range(len(mx)):
		for j in range(len(mx[0])):
			mx[i][j] /= x_sqrt[j]
			mx[i][j] *= w[j]
	x_max = []
	x_min = []
	for i in range(len(mx[0])):
		vec = []
		for x in mx:
			vec.append(x[i])
		x_max.append(max(vec))
		x_min.append(min(vec))
	d1 = []
	d2 = []
	for i in range(len(mx)):
		a = b = 0
		for j in range(len(mx[0])):
			a += (x_min[j] - mx[i][j]) ** 2
			b += (x_max[j] - mx[i][j]) ** 2
		d1.append(math.sqrt(a))
		d2.append(math.sqrt(b))
	res = []
	for i in range(len(d1)):
		res.append(round(d1[i] / (d1[i] + d2[i]), 3))
	return res
\end{lstlisting}

\section{Сравнение полученных результатов}
В результате использования методов, получены результаты, отраженные
в таблицах~\ref{table:saw:result}\,-\,\ref{table:topsis:result}.

\begin{table}[h!tp]
    \centering
    \caption{\leftline{Результаты метода SAW}}
    \label{table:saw:result}
    \begin{tabular}{|l|r|r|}
        \hline Видеоредактор & Коэффициент & Ранг\\ \hline
        VSDC & 0.639 & 1\\ \hline
        Kapwing & 0.471 & 8\\ \hline
        iMovie & 0.558 & 3\\ \hline
        Kdenlive & 0.493 & 7\\ \hline
        Lightworks & 0.423 & 10\\ \hline
        Avid Media Composer First & 0.621 & 2 \\ \hline
        Openshot & 0.464 & 9\\ \hline
        Clipchamp First & 0.555 & 4 \\ \hline
        Videopad & 0.493  & 6\\ \hline
        Davinci Resolve & 0.521 & 5 \\ \hline
    \end{tabular}
\end{table}

\begin{table}[h!tp]
    \centering
    \caption{\leftline{Результаты метода MOUT}}
    \label{table:mout:result}
    \begin{tabular}{|l|r|r|}
        \hline Видеоредактор & Коэффициент & Ранг\\ \hline
        VSDC & 0.632 & 1\\ \hline
        Kapwing & 0.458 & 8\\ \hline
        iMovie & 0.552 & 3\\ \hline
        Kdenlive & 0.478 & 6\\ \hline
        Lightworks & 0.408 & 10\\ \hline
        Avid Media Composer First & 0.625 & 2 \\ \hline
        Openshot & 0.428 & 9\\ \hline
        Clipchamp First & 0.546 & 4 \\ \hline
        Videopad & 0.476  & 7\\ \hline
        Davinci Resolve & 0.49 & 5 \\ \hline
    \end{tabular}
\end{table}

\begin{table}[h!tp]
    \centering
    \caption{\leftline{Результаты метода TOPSIS}}
    \label{table:topsis:result}
    \begin{tabular}{|l|r|r|}
        \hline Видеоредактор & Коэффициент & Ранг\\ \hline
        VSDC & 0.556 & 2\\ \hline
        Kapwing & 0.46 & 8\\ \hline
        iMovie & 0.538 & 3\\ \hline
        Kdenlive & 0.432 & 10\\ \hline
        Lightworks & 0.493 & 6\\ \hline
        Avid Media Composer First & 0.617 & 1 \\ \hline
        Openshot & 0.457 & 9\\ \hline
        Clipchamp First & 0.513 & 5 \\ \hline
        Videopad & 0.473  & 7\\ \hline
        Davinci Resolve & 0.534 & 4 \\ \hline
    \end{tabular}
\end{table}

\section{Выбор варианта программного обеспечения}
Опираясь на данные, полученные в ходе выполнения практической
работы, был выбран текстовый редактор VSDC.

\clearpage

\section*{\LARGE Заключение}
\addcontentsline{toc}{section}{Заключение}
В ходе выполнения практической работы были изучены основные характеристики и
атрибуты программного обеспечения. Также были изучены и
автоматизированы методы SAW, MOUT и TOPSIS. После автоматизации
опираясь на данные, полученные в ходе выполнения практической работы,
было выбрано наиболее оптимальное ПО.

\newpage

\begin{thebibliography}{0}
    \bibitem{Koch:saw:topsis}
    АНАЛИЗ МНОГОКРИТЕРИАЛЬНЫХ МЕТОДОВ ПРИНЯТИЯ
    УПРАВЛЕНЧЕСКИХ РЕШЕНИЙ
    (НА ПРИМЕРЕ ЗАДАЧИ ВЫБОРА ПОСТАВЩИКОВ
    МАТЕРИАЛЬНО-ТЕХНИЧЕСКИХ РЕСУРСОВ) /
    КОЧКИНА М.В., КАРАМЫШЕВ А.Н., МАХМУТОВ И.И.,
    ИСАВНИН А.Г., РОЗЕНЦВАЙГ А.К. // Набережные Челны, 2017 г.
    [Электронные ресурс]. URL:
    \url{https://kpfu.ru/staff_files/F633808541/Analiz_mnogokriterialnyh_metodov.pdf}.
    Дата обращения 04.03.2023.
    \bibitem{studfile:mout} Многокритериальная теория полезности (maut)
    [Электронные ресурс]. URL:
    \url{https://studfile.net/preview/5316711/page:24/}
    \bibitem{studfile:mout} Многокритериальная теория полезности (maut)
\end{thebibliography}