\graphicspath{{./img}} % path to graphics

\section*{Выполнение практической работы}
\addcontentsline{toc}{section}{Выполнение практической работы}
Дискретно-событийное моделирование --- почти такой же старый метод,
как системная динамика. В октябре 1961 года инженер компании IBM
Джеффри Гордон представил первую версию языка моделирования GPSS
(General Purpose Simulation System --- система моделирования общего
назначения, первоначальное название --- Gordon’s Programmable Simulation
System --- система программного моделирования Гордона). Это считается
первым случаем создания ПО для дискретно-событийного моделирования
(Borshchev 2013). В настоящее время многие программные средства
поддерживают этот метод. С точки зрения практического применения, мы
должны отметить, что более 50\% создаваемых имитационных моделей
разрабатываются с применением дискретно-событийного метода или
содержат компоненты, созданные с его помощью.\par
Основная идея дискретно-событийного моделирования заключается в том,
что система рассматривается как процесс, а именно как последовательность
операций, которые выполняются с разными структурными единицами
(заявками).\par
Прежде всего, нам
понадобится блок-источник для наших требований и блок "<выход из
системы"> для разработанного программного обеспечения. Когда у нас
появились требования, нам необходимо привлечь (осуществить захват
ресурса) разработчика из \texttt{опытного персонала}. Получив разработчика,
мы обрабатываем часть требований и отпускаем его (рис.~\ref{fig:model}).

\begin{image}
	\includegrph{1}
	\caption{Модель производства ПО}
	\label{fig:model}
\end{image}

Наиболее важная часть --- процесс поступления новых
разработчиков. Как наиболее важная часть процесса, она начинается с
блока-источника и заканчивается блоком \texttt{окончание\_обучения}
(Рис.~\ref{fig:model}).
Новые разработчики покидают блок \texttt{поступление персонала} согласно
\texttt{показателю найма} персонала в месяц. Когда появляется новый
разработчик, мы увеличиваем количество ресурсов в группы \texttt{новички}.
После этого новые сотрудники поступают в блок обслуживания обучение,
где сразу же приступают к обучению сроком в месяц. По окончании обучения
мы увеличиваем \texttt{опытный\_персонал}, добавив к нему новый ресурс, и
уменьшаем количество заявок-ресурсов в группе \texttt{новички}.\par
Запущенная модель проиллюстрированна на рисунке~\ref{fig:model:res}.

\begin{image}
	\includegrph{2}
	\caption{Запуск модели}
	\label{fig:model:res}
\end{image}

Так как модель создавалась в бесплатной версии Anylogic, есть
ограничения на 50000 прогонов, что не позволяет прогнать модель до конца
(до написания 500000 строк кода).

\clearpage

\section*{\LARGE Вывод}
\addcontentsline{toc}{section}{Вывод}
В данной практической работе была создана модель приозводства ПО,
исользуя метод --- дискретно-событийное моделирование.\par
Дискретно-событийное моделирование обычно используется для
технической составляющей управления проектами, когда менеджеры и
инженеры пытаются предсказать время
выполнения каждого этапа проекта и снизить риск
нарушения сроков по техническим причинам. По
сравнению с системной динамикой и агентным
моделированием дискретно-событийное
моделирование используется для низкого и
среднего уровня абстракции, когда детали
технологического или бизнес-процесса играют
ключевую роль.

